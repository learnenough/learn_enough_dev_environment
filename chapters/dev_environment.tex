One of the most important tasks for any aspiring developer---or any technical person generally---is setting up their computer as a \emph{development environment}, making it suitable for developing websites, web applications, and other software. \ledev is designed complement the \href{#}{main Learn Enough sequence} and the \rort\ by putting all (or almost all) of the relevant material in one place.

\section{Dev environment options} % (fold)
\label{sec:dev_environment_options}

Our focus in this tutorial is on installing or otherwise enabling the following four fundamental tools of software development (Figure~\ref{fig:dev_environment}):
\begin{enumerate}
  \item Command line terminal (``shell'')
  \item Text editor
  \item Version control (Git)
  \item Programming languages (Ruby, etc.)
\end{enumerate}
For more information on these different types of software application, see \lecl, \lete, \leg, and \ler.

\begin{figure}
\begin{center}
\image{images/figures/dev_environment.png}
\end{center}
\caption{Typical elements of a dev environment.\label{fig:dev_environment}}
\end{figure}

When setting up a development environment, there are three different possibilities we recommend, listed in increasing order of difficulty:

\begin{enumerate}
  \item Cloud \href{https://en.wikipedia.org/wiki/Integrated_development_environment}{IDE}
  \item Virtual Machine (VM)
  \item Native OS (macOS, Linux, Windows)
\end{enumerate}

If you're relatively inexperienced, we recommend starting with the cloud IDE (Section~\ref{sec:cloud_ide}), as it has the least difficult setup process. The VM option is also relatively straightforward in principle (Section~\ref{sec:virtual_machine}), although the relevant files are large and can lead to confusing errors when trying to install. Some may find also that the user interface (UI) isn't as polished as their native system.

\subsubsection{Native system} % (fold)
\label{sec:native_system}

While the IDE and VM options are great when you're just getting started, eventually it's important to be able to develop software on your native operating system (OS). Unfortunately, setting up a fully functional native development environment can be a challenging and frustrating process\footnote{This is why it's such a bad idea to include a native dev setup at the beginning of a book like the \rort. It's better to get started with the core material first, and tackle the native dev environment setup later. The switch over to Cloud9 in the third edition of the \rort\ was motivated by this realization.}---likely leaving ample opportunity to exercise your technical sophistication (Box~\ref{aside:technical_sophistication})---but it is an essential rite of passage for every aspiring technical wizard.

In order to tackle this difficult challenge, in Section~\ref{sec:native_os_setup} we'll discuss native OS setup for macOS, Linux, and Windows. As a warning, we'd like to note that the Windows section is currently anticlimactic, as it simply defers to the IDE and VM solutions, but please \href{mailto:support@learnenough.com}{let us know} if you'd like to help by contributing a truly native Windows solution.

\begin{aside}
\label{aside:technical_sophistication}
\heading{Technical sophistication.}

Lorem ipsum dolor sit amet, consectetur adipisicing elit, sed do eiusmod
tempor incididunt ut labore et dolore magna aliqua. Ut enim ad minim veniam,
quis nostrud exercitation ullamco laboris nisi ut aliquip ex ea commodo
consequat. Duis aute irure dolor in reprehenderit in voluptate velit esse
cillum dolore eu fugiat nulla pariatur. Excepteur sint occaecat cupidatat non
proident, sunt in culpa qui officia deserunt mollit anim id est laborum.

\end{aside}

% subsubsection native_system (end)

% section dev_environment_options (end)

\section{Cloud IDE}
\label{sec:cloud_ide}

The easiest dev environment option is a \emph{cloud IDE}, which is an integrated development environment in the \href{https://en.wikipedia.org/wiki/Cloud_computing}{cloud} that you access using a web browser of your choice. Although easy to activate, the resulting system is an industrial-grade (\href{https://www.ubuntu.com/}{Ubuntu} Linux) machine, not a toy. In addition, the cloud IDE automatically works cross-platform, since all you need is an ordinary web browser, which ever major OS provides.

There are several commercial options for running a cloud IDE, but as part of developing the \rort\ we partnered with \href{http://c9.io/}{Cloud9} (now part of Amazon Web Services). The resulting custom system, or ``workspace'', is appropriate for Ruby on Rails web development, and as a matter of course includes all the elements mentioned in Section~\ref{sec:dev_environment_options}. In particular, Cloud9 comes equipped with a command-line terminal and a text editor (including a filesystem navigator), as shown in Figure~\ref{fig:ide_anatomy}. Because each Cloud9 workspace provides a full working Linux system, it also automatically includes Git and Ruby (among other languages).

\begin{figure}
\begin{center}
\image{images/figures/ide_anatomy.png}
\end{center}
\caption{The anatomy of the cloud IDE.\label{fig:ide_anatomy}}
\end{figure}

If you decide to go the cloud IDE route, you can get started as follows:
\begin{enumerate}
\item \href{https://c9.io/web/sign-up/free}{Sign up for a free account at Cloud9}.\footnote{c9.io/signup} In order to prevent abuse, Cloud9 requires a valid credit card for signup, but the Rails Tutorial workspace is 100\% free, and your card will not be charged.
\item Click on ``Go to your Dashboard''.
\item Select ``Create New Workspace''.
\item Using the ``Rails Tutorial'' template, create a workspace with a name of your choice. We recommend using ``rails-tutorial'', but other names would work as well. It's important to avoid using underscores, which can cause problems when trying to access apps under development (i.e., don't use the name ``rails\_tutorial'').
\item Click ``Create workspace''.
\item After Cloud9 has finished provisioning the workspace, it should start automatically.
\end{enumerate}

\begin{figure}
\begin{center}
\image{images/figures/cloud9_new_workspace_4th_ed.png}
\end{center}
\caption{Creating a new workspace at Cloud9.\label{fig:cloud9_new_workspace}}
\end{figure}

If you plan to do any Ruby development, we also recommend changing the editor to use two spaces instead of the default four. As shown in Figure~\ref{fig:cloud9_two_spaces}, you can do this by clicking the gear icon in the upper right and then selecting ``Code Editor (Ace)'' to edit the ``Soft Tabs'' setting. (Note that this takes effect immediately; you don't need to click a ``Save'' button.)

\begin{figure}
\begin{center}
\image{images/figures/cloud9_two_spaces.png}
\end{center}
\caption{Setting Cloud9 to use two spaces for indentation.\label{fig:cloud9_two_spaces}}
\end{figure}

At this point, you're done! Although Internet access is required to use Cloud9 (which can be inconvenient when bandwidth options are limited), there is no alternative that combines so much power with such an easy setup.


\section{Virtual machine} % (fold)
\label{sec:virtual_machine}

A second option for setting up a development environment is a \emph{virtual machine}, or \emph{VM}, which is a fully functional computer system that runs inside the host system. In other words, you can run a full Linux system right inside of macOS or Windows.

The virtual option we recommend is one we developed as part of \lecl, which involves installing a Linux virtual machine on your native system. The steps appear as follows:

\begin{enumerate}
\item Install the right version of \href{https://www.virtualbox.org/}{VirtualBox} for your system (free).
\item Download the \href{https://softcover-static.s3.amazonaws.com/LearnEnough-v.1.4.ova}{Learn Enough Virtual Machine} (large file).
\item Once the download is complete, double-click the resulting ``OVA'' file and follow the instructions to install the Virtual Machine (VM).
\item Double-click the VM itself and log in using the default user's password, which is ``\texttt{foobar!}''.
\end{enumerate}
(Getting all these steps to work is a good exercise in technical sophistication (Box~\ref{aside:technical_sophistication}).)

The result of the steps above is a Linux desktop environment (Figure~\ref{fig:virtual_machine}) that comes equipped with all the elements mentioned in Section~\ref{sec:dev_environment_options}, including a command-line terminal, the Atom text editor, Git, Ruby, Python, and Node.js. The interface might not be as familiar, fast, or polished as your native OS, but the resulting development environment is industrial-strength and relatively easy to set up.

\begin{figure}
\begin{center}
\image{images/figures/virtual_machine.png}
\end{center}
\caption{A Linux virtual machine running inside a host OS.\label{fig:virtual_machine}}
\end{figure}

% section virtual_machine (end)


\section{Native OS setup} % (fold)
\label{sec:native_os_setup}

\subsection{macOS} % (fold)
\label{sec:macos}

\begin{itemize}
  \item \href{https://www.iterm2.com/downloads.html}{iTerm}
  \item \href{https://atom.io/}{Atom} editor
\end{itemize}


\subsubsection{Xcode command line tools}

\label{sec:shiny_xcode}

Xcode is a large suite of development tools and code libraries created by Apple, and it is a requirement for doing the kind of development covered by this tutorial. Thankfully, Apple has recently made Xcode incredibly quick and easy to install---it used to require a 4+ GB download of installation source files.

To install Xcode, open up your terminal and paste the following command in

\begin{codelisting}
\label{code:xcode-install}
\codecaption{Installing Xcode command line tools.}
%= lang:console
\begin{code}
$ xcode-select --install
\end{code}
\end{codelisting}

You'll be prompted by macOS to confirm that you want to install Xcode, do that and that's it!


\subsubsection{Homebrew}
\label{sec:homebrew}

Homebrew is a command line based package manager for macOS\@. If the phrase ``package manager'' isn't familiar to you, you can think of it as an application that works like an App Store---only it is filled with free open-source software.

If by chance you have played around with Linux in the past then you might have used a package manager in the past to install applications and utilities. Even though macOS is built on a similar foundation as Linux, Apple decided not to include a built in package manager to let you easily install software. Homebrew is one of many managers that is available to the open-source community, but over time it has become one of the most popular options in the Ruby development world.

bootstrap with system ruby

Installation of Homebrew is simple. First you are going to need to run a command from the terminal that is going to connect to the Homebrew repository, download it, and install it:

\begin{codelisting}
\label{code:homebrew-install}
\codecaption{Installing the Homebrew package manager.}
%= lang:console
\begin{code}
$ /usr/bin/ruby -e \
> "$(curl -fsSL https://raw.githubusercontent.com/Homebrew/install/master/install)"
\end{code}
\end{codelisting}

Hit enter when prompted to start the installation, and after everything finishes downloading and installing, it will suggest that you run this command to finish the installation. Do it!

%= lang:console
\begin{code}
$ brew doctor
\end{code}
The \kode{brew doctor} command starts a process that ensures that all of the directories and permissions needed by Homebrew to manage local files are corrected set up. If you have any problems at this point, you'll need to refer to the \href{https://github.com/Homebrew/homebrew/wiki/troubleshooting}{Homebrew troubleshooting wiki} (you really shouldn't though unless you've been making changes to random system folders and permissions).

\subsubsection{Git} % (fold)
\label{sec:git}

Should come with Xcode command-line tools. Can check

%= lang:console
\begin{code}
$ which git
\end{code}

If not there, can install with Homebrew:

%= lang:console
\begin{code}
$ brew install git
\end{code}

% subsubsection git (end)

\subsubsection{Node} % (fold)
\label{sec:node}

%= lang:console
\begin{code}
$ brew install nodejs
\end{code}

% subsubsection node (end)


\subsubsection{Ruby Environment (rbenv)}
\label{sec:rbenv}

Now we are starting to get to the core of the development environment! Rbenv is a utility that will run on your computer to manage any versions of Ruby you install and it will make sure that the gems (plug-ins) you install are placed in the right spot for Ruby to find. The rbenv system is modular and allows you to specify a different version of Ruby (and the associated gems) for different project repositories.

The full level of functionality isn't really needed for this project, but if you continue to do web development you will find that you need to lock certain projects to certain versions of Ruby because of dependencies that will only work on a specific version of Ruby. That could cause you to be unable to use your local development environment without going through some annoying updating and reconfiguring of individual applications.

To get started, from the terminal run:

%= lang:console
\begin{code}
$ brew install rbenv ruby-build
\end{code}

When the download and installation finishes, run the following at the command line to get rbenv up and running:

%= lang:console
\begin{code}
$ rbenv init
\end{code}
Now, if you are like me, you aren't going to want to have to think about starting up a management utility like rbenv every time that you open up your terminal. To set your system to always be ready for development, you are going to need to add a command to your \kode{bash\_profile}. If you completed the Learn Enough Text Editor tutorial then you are already familiar with editing the \kode{bash\_profile}, if you didn't do that tutorial and / or don't know if you have a \kode{bash\_profile} created, run this command in your terminal:

%= lang:console
\begin{code}
$ touch ~/.bash_profile
\end{code}

Right now we are only going to add a single line to it, if your computer already has a \emph{bash\_profile} from previous work, you can just open it up and check to see if the \kode{eval "\$(rbenv init -)"} line is already in it. Otherwise, to both create the file and / or add the command to start rbenv, run this in the terminal:

%= lang:console
\begin{code}
$ echo 'eval "$(rbenv init -)"' >> ~/.bash_profile
\end{code}

If you want to make sure that everything was added correctly, you can run this in the terminal:

%= lang:console
\begin{code}
$ cat ~/.bash_profile
\end{code}

That will spit out the contents for the \kode{bash\_profile} into the terminal window, and if you see \kode{eval "\$(rbenv init -)"} everything is correct. The last thing that you need to do is to let the current terminal window that you are working in know that there is a new \kode{bash\_profile} with configurations to load.

%= lang:console
\begin{code}
$ source ~/.bash_profile
\end{code}
You now have a Ruby environment manager that is ready to handle custom versions of Ruby and keep track of any gems that you install for each version.

\subsubsection{New Ruby version}
\label{sec:install_ruby}

Now that your environment manager is set up, let's give it a non-system version of Ruby to manage. The installation process is handled entirely by rbenv, so all you have to do is instruct it as to which version you'd like on your system by passing along the exact Ruby version and patch version number like this:

\begin{codelisting}
\label{code:ruby-nstall}
\codecaption{Installing a fresh copy of Ruby.}
%= lang:console
\begin{code}
$ rbenv install 2.1.3
\end{code}
\end{codelisting}

You will see rbenv start the download process and install any dependencies that are needed for that specific verion of Ruby. When the installation finishes, run:

%= lang:console
\begin{code}
$ rbenv rehash
\end{code}
The \kode{rehash} command lets rbenv know that there is a new version of Ruby on the system that it needs to make available for you to use.

For this guide, we are also going to make the 2.1.3 version the global default so that you won't have to worry about specifying the Ruby version when you start your project. To make the Ruby version you just installed into the default, run this in your terminal window:

%= lang:console
\begin{code}
$ rbenv global 2.1.3
\end{code}


% subsection macos (end)

\subsection{Linux} % (fold)
\label{sec:linux}

Higher level of tech soph assumed

Already includes terminal, \href{https://git-scm.com/download/linux}{Git installation}

Atom

Follow \href{https://github.com/rbenv/rbenv#installation}{rbenv installation} instructions from the rbenv website. It includes more permutations

% subsection linux (end)

\subsection{Windows} % (fold)
\label{sec:windows}

Use Cloud9, VM. Open to suggestions!

% subsection windows (end)

% section native_os_setup (end)

\section{Conclusion} % (fold)
\label{sec:conclusion}

Enough to be dangerous. Ready to get going on other things.

% section conclusion (end)