One of the most important tasks for any aspiring developer---or any technical person generally---is setting up their computer as a \emph{development environment}, making it suitable for developing websites, web applications, and other software. \ledev is designed to complement the \href{#}{main Learn Enough sequence} and the \rort\ by putting all (or almost all) of the relevant material in one place.

\section{Dev environment options} % (fold)
\label{sec:dev_environment_options}

Our focus in this tutorial is on installing or otherwise enabling the following four fundamental tools of software development (Figure~\ref{fig:dev_environment}):
\begin{enumerate}
  \item Command line terminal (``shell'')
  \item Text editor
  \item Version control (Git)
  \item Programming languages (Ruby, etc.)
\end{enumerate}
For more information on these different types of software application, see \lecl, \lete, \leg, and \ler.

\begin{figure}
\begin{center}
\image{images/figures/dev_environment.png}
\end{center}
\caption{Typical elements of a dev environment.\label{fig:dev_environment}}
\end{figure}

When setting up a development environment, there are three different possibilities we recommend, listed in increasing order of difficulty:

\begin{enumerate}
  \item Cloud \href{https://en.wikipedia.org/wiki/Integrated_development_environment}{IDE}
  \item Virtual Machine (VM)
  \item Native OS (macOS, Linux, Windows)
\end{enumerate}

If you're relatively inexperienced, we recommend starting with the cloud IDE (Section~\ref{sec:cloud_ide}), as it has the least difficult setup process. The VM option is also relatively straightforward in principle (Section~\ref{sec:virtual_machine}), although the relevant files are large and can lead to confusing errors when trying to install. Some may find also that the user interface (UI) isn't as polished as their native system.

\subsubsection{Native system} % (fold)
\label{sec:native_system}

While the IDE and VM options are great when you're just getting started, eventually it's important to be able to develop software on your native operating system (OS). Unfortunately, setting up a fully functional native development environment can be a challenging and frustrating process\footnote{This is why it's such a bad idea to include a native dev setup at the beginning of a book like the \rort. It's better to get started with the core material first, and tackle the native dev environment setup later. The switch over to Cloud9 in the third edition of the \rort\ was motivated by this realization.}---likely leaving ample opportunity to exercise your technical sophistication (Box~\ref{aside:technical_sophistication})---but it is an essential rite of passage for every aspiring technical wizard.

In order to tackle this difficult challenge, in Section~\ref{sec:native_os_setup} we'll discuss native OS setup for macOS, Linux, and Windows. As a warning, we'd like to note that the Windows section is currently anticlimactic, as it simply defers to the IDE and VM solutions, but please \href{mailto:support@learnenough.com}{let us know} if you'd like to help by contributing a truly native Windows solution.

***should only do if finished \lecl.

\begin{aside}
\label{aside:technical_sophistication}
\heading{Technical sophistication.}

Lorem ipsum dolor sit amet, consectetur adipisicing elit, sed do eiusmod
tempor incididunt ut labore et dolore magna aliqua. Ut enim ad minim veniam,
quis nostrud exercitation ullamco laboris nisi ut aliquip ex ea commodo
consequat. Duis aute irure dolor in reprehenderit in voluptate velit esse
cillum dolore eu fugiat nulla pariatur. Excepteur sint occaecat cupidatat non
proident, sunt in culpa qui officia deserunt mollit anim id est laborum.

\end{aside}

% subsubsection native_system (end)

% section dev_environment_options (end)

\section{Cloud IDE}
\label{sec:cloud_ide}

The easiest dev environment option is a \emph{cloud IDE}, which is an integrated development environment in the \href{https://en.wikipedia.org/wiki/Cloud_computing}{cloud} that you access using a web browser of your choice. Although easy to activate, the resulting system is an industrial-grade (\href{https://www.ubuntu.com/}{Ubuntu} Linux) machine, not a toy. In addition, the cloud IDE automatically works cross-platform, since all you need is an ordinary web browser, which ever major OS provides.

There are several commercial options for running a cloud IDE, but as part of developing the \rort\ we partnered with \href{http://c9.io/}{Cloud9} (now part of Amazon Web Services). The resulting custom system, or ``workspace'', is appropriate for Ruby on Rails web development, and as a matter of course includes all the elements mentioned in Section~\ref{sec:dev_environment_options}. In particular, Cloud9 comes equipped with a command-line terminal and a text editor (including a filesystem navigator), as shown in Figure~\ref{fig:ide_anatomy}. Because each Cloud9 workspace provides a full working Linux system, it also automatically includes Git and Ruby (among other languages).

\begin{figure}
\begin{center}
\image{images/figures/ide_anatomy.png}
\end{center}
\caption{The anatomy of the cloud IDE.\label{fig:ide_anatomy}}
\end{figure}

If you decide to go the cloud IDE route, you can get started as follows:
\begin{enumerate}
\item \href{https://c9.io/web/sign-up/free}{Sign up for a free account at Cloud9}.\footnote{c9.io/signup} In order to prevent abuse, Cloud9 requires a valid credit card for signup, but the Rails Tutorial workspace is 100\% free, and your card will not be charged.
\item Click on ``Go to your Dashboard''.
\item Select ``Create New Workspace''.
\item Using the ``Rails Tutorial'' template, create a workspace with a name of your choice. We recommend using ``rails-tutorial'', but other names would work as well. It's important to avoid using underscores, which can cause problems when trying to access apps under development (i.e., don't use the name ``rails\_tutorial'').
\item Click ``Create workspace''.
\item After Cloud9 has finished provisioning the workspace, it should start automatically.
\end{enumerate}

\begin{figure}
\begin{center}
\image{images/figures/cloud9_new_workspace_4th_ed.png}
\end{center}
\caption{Creating a new workspace at Cloud9.\label{fig:cloud9_new_workspace}}
\end{figure}

If you plan to do any Ruby development, we also recommend changing the editor to use two spaces instead of the default four. As shown in Figure~\ref{fig:cloud9_two_spaces}, you can do this by clicking the gear icon in the upper right and then selecting ``Code Editor (Ace)'' to edit the ``Soft Tabs'' setting. (Note that this takes effect immediately; you don't need to click a ``Save'' button.)

\begin{figure}
\begin{center}
\image{images/figures/cloud9_two_spaces.png}
\end{center}
\caption{Setting Cloud9 to use two spaces for indentation.\label{fig:cloud9_two_spaces}}
\end{figure}

At this point, you're done! Although Internet access is required to use Cloud9 (which can be inconvenient when bandwidth options are limited), there is no alternative that combines so much power with such an easy setup.


\section{Virtual machine} % (fold)
\label{sec:virtual_machine}

A second option for setting up a development environment is a \emph{virtual machine}, or \emph{VM}, which is a fully functional computer system that runs inside the host system. In the case of the Learn Enough VM, you can run a full Linux system right inside of macOS or Windows (or even Linux! \href{https://en.wikipedia.org/wiki/Turtles_all_the_way_down}{It's turtles all the way down} (Figure~\ref{fig:turtles}).\footnote{Image retrieved from https://upload.wikimedia.org/wikipedia/commons/4/47/River\_terrapin.jpg on 2017-01-24 and is in the public domain.}).

\begin{figure}
\begin{center}
\image{images/figures/turtles.jpg}
\end{center}
\caption{It's turtles all the way down.\label{fig:turtles}}
\end{figure}

The virtual option we recommend is one we developed as part of \lecl, which involves installing a Linux virtual machine on your native system. The steps appear as follows:

\begin{enumerate}
\item Install the right version of \href{https://www.virtualbox.org/}{VirtualBox} for your system (free).
\item Download the \href{https://softcover-static.s3.amazonaws.com/LearnEnough-v.1.4.ova}{Learn Enough Virtual Machine} (large file).
\item Once the download is complete, double-click the resulting ``OVA'' file and follow the instructions to install the Virtual Machine (VM).
\item Double-click the VM itself and log in using the default user's password, which is ``\texttt{foobar!}''.
\end{enumerate}
(Getting all these steps to work is a good exercise in technical sophistication (Box~\ref{aside:technical_sophistication}).)

The result of the steps above is a Linux desktop environment (Figure~\ref{fig:virtual_machine}) that comes equipped with all the elements mentioned in Section~\ref{sec:dev_environment_options}, including a command-line terminal, the Atom text editor, Git, and Ruby. The interface might not be as familiar, fast, or as polished as your native OS, but the resulting development environment is industrial-strength and relatively easy to set up.

\begin{figure}
\begin{center}
\image{images/figures/virtual_machine.png}
\end{center}
\caption{A Linux virtual machine running inside a host OS.\label{fig:virtual_machine}}
\end{figure}

% section virtual_machine (end)


\section{Native OS setup} % (fold)
\label{sec:native_os_setup}

As mentioned in Section~\ref{sec:dev_environment_options}, setting up your native operating system as a development environment can be challenging, but is an important step to take once you've reached a certain level of technical sophistication. The cloud IDE or virtual machine options are great places to start, but eventually you have to grab the bull by the horns and bend your native system to your will (Figure~\ref{fig:grab_bull_by_horns}).\footnote{Image retrieved from https://www.flickr.com/photos/mikey\_loves\_bcn/4354275361 on 2017-01-24. Copyright © 2010 by \href{https://www.flickr.com/photos/mikey_loves_bcn/}{Mikey V} and used under the terms of the \ccbync\ license.}

\begin{figure}
\begin{center}
\image{images/figures/grab_bull_by_horns.jpg}
\end{center}
\caption{Sometimes you have to grab the bull by the horns.\label{fig:grab_bull_by_horns}}
\end{figure}

\subsection{macOS} % (fold)
\label{sec:macos}

The native Macintosh operating system, formerly called Mac OS~X and now known as \emph{macOS}, has a polished graphical user interface (GUI) while also being based on Unix. As a result, macOS is ideally suited for use as a  programmer's development environment.

The steps in this section constitute more than just a minimal system; you can actually get away with doing a lot less, but your three authors all use macOS, and we feel it's important not to shortchange you with a half-baked setup.

\subsubsection{Terminal and editor} % (fold)
\label{sec:terminal_and_editor}

Although macOS comes with a native terminal program, we recommend you \href{https://www.iterm2.com/downloads.html}{install iTerm}, which includes \href{https://www.iterm2.com/features.html}{various enhancements} that make it a better choice for developers and other technical users.

We also recommend installing a programmer's text editor. There are lots of excellent choices, but the \href{https://atom.io/}{Atom editor} (covered in \lete) is a good place to start if you don't already have a favorite.

% subsubsection terminal_and_editor (end)


\subsubsection{Xcode command line tools}
\label{sec:shiny_xcode}

Although based on Unix,\footnote{Specifically, the \href{https://en.wikipedia.org/wiki/NeXT}{NeXT} system developed by the company Steve Jobs founded in 1985 after being ousted from Apple. The NeXT OS became the foundation for Mac OS X (later, macOS) after Apple acquired NeXT in 1997, which also led to Jobs returning as Apple CEO.} macOS doesn't ship with all the software necessary for a proper development environment. In order to fill this gap, macOS users should install \emph{Xcode}, a large suite of development tools and code libraries created by Apple.

Xcode used to require a 4+ GB download of installation source files, but thankfully Apple has recently made Xcode incredibly quick and easy to install with a simple command-line command, as shown in Listing~\ref{code:xcode-install}.

\begin{codelisting}
\label{code:xcode-install}
\codecaption{Installing Xcode command line tools.}
%= lang:console
\begin{code}
$ xcode-select --install
\end{code}
\end{codelisting}


\subsubsection{Homebrew}
\label{sec:homebrew}

The next step is technically optional but in our view is necessary for a truly professional-grade macOS dev environment: namely, installing the outstanding \emph{Homebrew} package manager.

You can think of a package manager as an App Store that runs at the command line and is filled with free open-source software. Nowadays most Linux distributions come with a native package manager (Section~\ref{sec:linux}), but by default macOS doesn't come with one. Homebrew is one of many managers that is available to the open-source community, but over time it has become the most popular options among serious macOS developers.

Installing Homebrew requires the Ruby programming language, which might seem like a problem since we haven't yet installed Ruby. Happily, macOS ships with a \emph{system Ruby} that we can use to \href{https://en.wikipedia.org/wiki/Bootstrapping}{bootstrap} the installation. We'll use this default Ruby to install Homebrew, and then in Section~\ref{sec:rbenv} we'll use Homebrew to install the Ruby environment manager \emph{rbenv}, which in turn will allow us to install additional Ruby versions of our choosing.

The Homebrew installation program, or \emph{script}, is currently hosted at GitHub, and can be accessed using the \kode{curl} program \href{https://www.learnenough.com/command-line-tutorial#sec-downloading_a_file}{covered} in \lecl. We can install Homebrew by running this install script using the system Ruby program, which is located in \kode{/usr/bin/ruby}.

The full command appears as in Listing~\ref{code:homebrew-install}. Note that you shouldn't include the prompt~\kode{\$} or the line-continuation indicator \kode{>}, but you should include the backslash \kode{\textbackslash}, which allows us to break longer lines into multiple parts. You can also just copy and paste the full command from the \href{http://brew.sh/}{Homebrew home page} if you like. (The line break is only included to fit the design constraints of this document.)

\begin{codelisting}
\label{code:homebrew-install}
\codecaption{Installing the Homebrew package manager.}
%= lang:console
\begin{code}
$ /usr/bin/ruby -e \
> "$(curl -fsSL https://raw.githubusercontent.com/Homebrew/install/master/install)"
\end{code}
\end{codelisting}

Homebrew installs the \kode{brew} command-line command for installing, updating, and removing packages. After the installation of Homebrew finishes, you will be prompted to run \kode{brew doctor}, which ensures that all of the directories and permissions needed by Homebrew to manage local files are correctly set up:

%= lang:console
\begin{code}
$ brew doctor
\end{code}

\noindent If you have any problems at this point, you'll need to refer to the \href{https://github.com/Homebrew/homebrew/wiki/troubleshooting}{Homebrew troubleshooting wiki}, but you really shouldn't though unless you've been making changes to random system folders and permissions.

\subsubsection{Git} % (fold)
\label{sec:git}

A recent version of the Git version control system should come automatically with Xcode command-line tools installed in Section~\ref{sec:shiny_xcode}. You can verify this using the \kode{which} command:

%= lang:console
\begin{code}
$ which git
\end{code}

\noindent If the result of this is blank, it means Git isn't installed, and you can install it with Homebrew:

%= lang:console
\begin{code}
$ brew install git    # only if `which git` is blank
\end{code}

% subsubsection git (end)


\subsubsection{Ruby Environment (rbenv)}
\label{sec:rbenv}

Now that Homebrew and rbenv are installed, it's time to install a custom Ruby suitable for software (especially web) development. Although macOS comes with Ruby preinstalled (as we saw in Section~\ref{sec:homebrew}), as developers we don't have any control over the exact version, and macOS doesn't natively allow us to use multiple versions of Ruby in parallel.

To give us more flexibility with our development environment, we'll install \emph{rbenv}, which is a utility that manages different Ruby versions and makes sure that Ruby software packages (called \emph{gems}) (plug-ins) get placed in the right spot for Ruby to find. Using rbenv together with the associated \emph{ruby-build} project also allows us to specify a different version of Ruby (and the associated gems) for different project repositories.\footnote{The \href{http://docs.python-guide.org/en/latest/dev/virtualenvs/}{\emph{virtualenv}} utility accomplishes a similar task for Python projects.} This is a common task in software development. For example, an older version of a program might need an older version of Ruby to run correctly. Using rbenv means we can support such an older program while still running a more up-to-date version of Ruby for our other projects.

Installing rbenv is easy with Homebrew. We can install both rbenv and ruby-build, as shown in Listing~\ref{code:rbenv}.

\begin{codelisting}
\label{code:rbenv}
\codecaption{Installing rbenv.}
%= lang:console
\begin{code}
$ brew install rbenv ruby-build
\end{code}
\end{codelisting}

After the download and installation from Listing~\ref{code:rbenv} finishes, we need to get rbenv up and running using \kode{rbenv init}, as shown in Listing~\ref{code:rbenv_init}.

\begin{codelisting}
\label{code:rbenv_init}
\codecaption{Initializing rbenv.}
%= lang:console
\begin{code}
$ rbenv init
# Load rbenv automatically by appending
# the following to ~/.bash_profile:

eval "$(rbenv init -)"
\end{code}
\end{codelisting}

\noindent If Listing~\ref{code:rbenv_init} gives you an error message like ``No such file or directory.'', try exiting your shell program with \verb+⌃D+ and restarting it, and then try the command again. (This sort of restart-and-retry technique is classic technical sophistication (Box~\ref{aside:technical_sophistication}).)

As seen in Listing~\ref{code:rbenv_init}, running \kode{rbenv init} gives us a suggestion for how to avoid having to initializing rbenv by hand each time: we simply need to append the line

%= lang:bash
\begin{code}
eval "$(rbenv init -)"
\end{code}

\noindent to our Bash profile file \kode{.bash\_profile} (which is \href{https://www.learnenough.com/text-editor-tutorial#sec-saving_and_quitting_files}{covered} in \lete).

Of course, you can use a text editor to add the \kode{eval} line to \kode{.bash\_profile}, but the easiest way is to use \kode{echo} and the append operator~\kode{>{}>} \href{https://www.learnenough.com/command-line-tutorial#sec-redirecting_and_appending}{covered} in \lecl, like this:

%= lang:console
\begin{code}
$ echo 'eval "$(rbenv init -)"' >> ~/.bash_profile
\end{code}

\noindent Note that we've included the home directory \kode{\textasciitilde} in the path so that it works no matter which directory we're currently in.

Finally, to activate the new profile file we need to \kode{source} it (as \href{https://www.learnenough.com/text-editor-tutorial#code-source_command}{mentioned} in \lete):

%= lang:console
\begin{code}
$ source ~/.bash_profile
\end{code}


\subsubsection{New Ruby version}
\label{sec:install_ruby}

Now that rbenv is set up, let's give it a non-system version of Ruby to manage. The installation process is handled entirely by rbenv, so all you have to do is instruct it as to which version you'd like on your system by passing along the exact Ruby version number.

In case it's been a while since you installed rbenv and ruby-build, we'll first \kode{update} Homebrew and \kode{upgrade} rbenv and ruby-build, as follows:

%= lang:console
\begin{code}
$ brew update
$ brew upgrade rbenv ruby-build
\end{code}

\noindent If you already have the latest versions, as you will if you just completed Section~\ref{sec:rbenv}, you will get a couple of ``error'' messages, which you can safely ignore.

Now we're ready to install a custom version of Ruby. We'll use Ruby 2.3.3 in this tutorial, which as of this writing works with a wide variety of Ruby applications, but you can also use \href{https://www.ruby-lang.org/en/downloads/}{current stable version} as listed at the Ruby website.\footnote{At the moment, Ruby 2.4.0 is the current stable version, but its unification of the \kode{Fixnum} and \kode{Bignum} classes into a single \kode{Integer} class causes many older programs to emit verbose deprecation warnings, which is why right now we're going with Ruby 2.3.3.}

To install the desired version of Ruby using \kode{rbenv}, simply execute the command shown in Listing~\ref{code:ruby-install}.

\begin{codelisting}
\label{code:ruby-install}
\codecaption{Installing a fresh copy of Ruby.}
%= lang:console
\begin{code}
$ rbenv install 2.3.3
\end{code}
\end{codelisting}

\noindent After running the command in Listing~\ref{code:ruby-install}, you should see rbenv start the download process and install any dependencies that are needed for that specific version of Ruby (which might take a while depending on bandwidth and CPU limitations).

After the Ruby installation finishes, we need to tell the system that there's a new version of Ruby using the obscurely named \kode{rehash} command:

%= lang:console
\begin{code}
$ rbenv rehash
\end{code}

For this guide, we are also going to set the Ruby version from Listing~\ref{code:ruby-install} as the global default so that you won't have to worry about specifying the Ruby version when you start your project. The way to do this is with the \kode{global} command:

%= lang:console
\begin{code}
$ rbenv global 2.3.3
\end{code}

\noindent At this point, it's probably a good idea to restart your shell program to make sure all the settings are properly updated.

For future work, you may want to use specific versions of Ruby on a per-project basis, which can be done by creating a file called \kode{.ruby-version} in the project's root directory and including the version of Ruby to be used. (You'll also have to install it, of course, using \kode{rbenv install <version number>}.) See the \href{https://github.com/rbenv/rbenv}{rbenv documentation} for more information.

% subsection macos (end)

\subsection{Linux} % (fold)
\label{sec:linux}

Because of Linux's highly technical origins, Linux systems typically come well-equipped with developer tools. As a result, setting up a native Linux OS as a dev environment is especially simple.

Every major Linux distribution ships with a terminal program, a text editor, and Git. There are only three major steps we recommend in addition to the defaults:
\begin{enumerate}
  \item \href{https://atom.io/}{Download and install Atom} if you don't already have a favorite editor.
  \item Follow the \href{https://github.com/rbenv/rbenv#installation}{rbenv and ruby-build installation instructions} from the rbenv website.
  \item Install Ruby as shown in Listing~\ref{code:ruby-install}.
\end{enumerate}

At this point, you should be good to go!

% subsection linux (end)

\subsection{Windows} % (fold)
\label{sec:windows}

Finally, and anti-climactically, we have native (non-)instructions for Microsoft Windows. Because none of your humble authors \href{http://blog.dictionary.com/none/}{are} well-acquainted with Windows software development, we did a Twitter survey, and found that the typical response was ``Use a Linux VM\@.'' Sigh.

So, for now we officially recommend either a cloud IDE (Section~\ref{sec:cloud_ide}) or a Linux VM (Section~\ref{sec:virtual_machine}) if you're using Windows. We're definitely interested in a more natively appropriate solution, though, so please \href{mailto:support@learnenough.com}{let us know} if you'd like to volunteer suggestions. Thanks!

% subsection windows (end)

% section native_os_setup (end)

\section{Conclusion} % (fold)
\label{sec:conclusion}

If you've made it this far---and especially if you completed a native OS setup in Section~\ref{sec:native_os_setup}---you've now learned enough development environment to be \emph{dangerous}. You're ready to move on to complete challenging tutorials like \lecss, \ler, and the \rort. Good luck!

% section conclusion (end)

\noindent {\small \ledev. Copyright © 2017 by Michael Hartl, Lee Donahoe, and Nick Merwin.}
